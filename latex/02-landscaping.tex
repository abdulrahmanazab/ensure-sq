\subsection{Research Software stack AND types}

Research Software stack is taken from~\cite{hinsen2019}:

\begin{center}
    \tablehead{
        \hline
        \textbf{Stack} & \textbf{Stack definition} & \textbf{Types} \\
        \hline
    }
    \tabletail{\hline}
    \tablelasttail{\hline}
    
    \bottomcaption{RS stacks, stacks definitions and RS types. Stack level 1 is not considered in this report.}
    \label{tab:rs_stacks}

    \small
    \begin{supertabular}{|p{0.15\linewidth}|p{0.5\linewidth}|p{0.3\linewidth}|}

    \hline
    4 - Project specific code &
    Software written by scientists for a specific research project. It can take various forms including scripts, notebooks, and workflows, but also special-purpose libraries and utilities. &
    Library; Analysis script, workflows; Services and platforms \\ \hline

    3 - Domain specific tools &
    Tools and libraries that implement models and methods which are developed and used by specific communities. Gromacs, MMTK, Amber. &
    Library; Application (such as Monte-Carlo simulation); Services and platforms
    \\ \hline

    2 - Scientific infrastructure &
    Infrastructure created specifically for scientific computing, but not for any particular application domain: mathematical libraries such as BLAS, LAPACK, or SciPy, scientific data management tools such as HDF5. &
    Library; Framework; Services and platforms\\ \hline

    \it{1 - Non-scientific infrastructure} &
    \it{Compilers and interpreters, libraries for data management; gcc, python...} &
    -- \\ \hline

    \end{supertabular}
\end{center}

The RS types can be summarized next:

\begin{itemize}
\item Library
\item Framework
\item Application (such as Monte-Carlo simulation)
\item Analysis script
\item Services and platforms
\end{itemize}

\subsection{Definition and references of Research SW}

% \dg{[summary discussion] The expectation of research software is that it can support the findings in a paper. In that sense, RS is "expected" to
% reproduce those findings, while regular software would not necessarily have this expectation (as long as it works).}

The goal of this section is to cross-reference the elements of
software quality from the previous section with the SG1 typology of
research software development, in order to identify the tools and
infrastructures needed to satisfy minimum quality criteria. 

Discussion on definition of research software \url{https://zenodo.org/record/5504016#.ZGc6SnbP2Uk}

As a reminder, SG1 has defined different categories of software
development related to the context in which they are performed. Others have different categories as well, for instance \url{https://docs.google.com/presentation/d/1uwxSwd8chbG7bVn5lPvNNhv5f_JeeelA2mN9NXhlOhA/edit#slide=id.g183ddf12e8b_0_122} and \url{https://zenodo.org/record/7589725#.ZGc5E3bP2Uk}

\subsection{Software Quality models: Survey}

We follow the methodology for the survey, proposed by Kitchenham and Charters \cite{keele2007guidelines} which has the following steps:

\begin{enumerate}
    \item Source selection and search: We have searched in the Scopus dataset, including the top five journals in software engineering related to software \footnote{\url{https://research.com/journals-rankings/computer-science/software-programming}} and articles of the  ``International Conference on Software Engineering", one of the top venues for Software engineering. We also added documents and web resources that the Task Force subgroup considered relevants. The search included the keywords ``software quality" in the title of the target publications.
    \item Inclusion and exclusion criteria: Excluded journals not in the SE domain. Excluded articles not written in English.
    \item Selection procedure: Skim article titles and abstracts. The process was performed by 2-3 people. Final list was agreed upon by the group through discussion about the relevance of the paper and analysis if that paper contains or proposes Software Quality attributes.
    \item Review process: After following the selection procedure, we ended up with 19 articles, which have been reviewed in this survey. Some of the articles refer to the ISO/IEC 25010:2011(E)\cite{iso_25010_2011_2017} or to its precursor ISO/IEC 9126, have been grouped together.
\end{enumerate}

Journals: IEEE Transactions on Software Engineering, Empirical Software Engineering, Journal of Systems and Software, Software \& Systems Modeling, Information and Software Technology, IEEE Software, Software Quality Journal. Query used:

\tiny
\begin{verbatim}
    TITLE ( software  AND quality )  AND  
        ( LIMIT-TO ( EXACTSRCTITLE ,  "Software Quality Journal" )  
        OR  LIMIT-TO ( EXACTSRCTITLE ,  "Proceedings International Conference On Software Engineering" ) 
        OR  LIMIT-TO ( EXACTSRCTITLE ,  "IEEE Transactions on Software Engineering" )
        OR  LIMIT-TO ( EXACTSRCTITLE ,  "Empirical Software Engineering" ) 
        OR  LIMIT-TO ( EXACTSRCTITLE ,  "Journal of Systems and Software" ) 
        OR  LIMIT-TO ( EXACTSRCTITLE ,  "Software & Systems Modeling" ) 
        OR  LIMIT-TO ( EXACTSRCTITLE ,  "Information and Software Technology" )  
        OR  LIMIT-TO ( EXACTSRCTITLE ,  "IEEE Software" )   
        )  AND  ( LIMIT-TO ( SUBJAREA ,  "COMP" )  OR  LIMIT-TO ( SUBJAREA ,  "ENGI" ) )  
\end{verbatim}
\small

As a result, the sub group obtained 272 results. Additional filtering excluded:

\begin{itemize}
    \item Papers with no abstracts.
    \item Proceedings/workshop summary.
    \item Those which did not seem related by browsing the abstract and title.
    \item Papers that did not seem to propose quality dimensions (e.g., if they talk about practices).
\end{itemize}

There where 147 papers after filtering + 4 documents that were not published as paper but considered relevant by the sub group. In the end there were 19 articles that were reviewed by 2-3 persons each.

% Query: \footnote{\url{https://www.scopus.com/results/results.uri?sort=plf-f&src=s&nlo=&nlr=&nls=&sid=b04451f7b887660ce99d73bfdbdc4fc8&sot=a&sdt=a&cluster=scosubjabbr%2c%22COMP%22%2ct%2c%22ENGI%22%2ct%2bscoexactsrctitle%2c%22Software+Quality+Journal%22%2ct%2c%22Proceedings+International+Conference+On+Software+Engineering%22%2ct%2c%22IEEE+Transactions+on+Software+Engineering%22%2ct%2c%22Empirical+Software+Engineering%22%2ct%2c%22Journal+of+Systems+and+Software%22%2ct%2c%22Software+%26+Systems+Modeling%22%2ct%2c%22Information+and+Software+Technology%22%2ct%2c%22IEEE+Software%22%2ct&sl=30&s=TITLE+%28+software+AND+quality+%29&cl=t&offset=201&origin=resultslist&ss=plf-f&ws=r-f&ps=r-f&cs=r-f&cc=10&txGid=cff6749bf8a084912200dbef379950d4} 
% Also, this book: https://books.google.es/books?hl=en&lr=&id=XTvpAQAAQBAJ&oi=fnd&pg=PR3&dq=software+quality&ots=fohz_-KW0d&sig=5TGlvR3sgAIkAHzs5Iup8Qijpuo#v=onepage&q=software%20quality&f=false looks nice!
