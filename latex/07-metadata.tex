\subsection{Metadata schema for SW}

(Codemeta, other(s)...)

\subsection{The case for FAIR for Research Software (FAIR4RS)}

The FAIR principles for Research Software are stated in \cite{chue_hong_neil_p_2022_6623556}, they
are reproduced here for reference:

\textbf{F: Software, and its associated metadata, is easy for both humans and machines to find.}

\begin{itemize}
    \item \textbf{F1} - Software is assigned a globally unique and persistent identifier.
    \begin{itemize}
        \item \textbf{F1.1} - Components of the software representing levels of granularity are assigned distinct identifiers.
        \item \textbf{F1.2} - Different versions of the software are assigned distinct identifiers.
    \end{itemize}

    \item \textbf{F2} - Software is described with rich metadata.
    \item \textbf{F3} - Metadata clearly and explicitly include the identifier of the software they describe.
    \item \textbf{F4} - Metadata are FAIR, searchable and indexable.
\end{itemize}

\textbf{A: Software, and its metadata, is retrievable via standardized protocols.}

\begin{itemize}
    \item \textbf{A1} - Software is retrievable by its identifier using a standardized communications protocol.

    \begin{itemize}
        \item \textbf{A1.1} - The protocol is open, free, and universally implementable.
        \item \textbf{A1.2} - The protocol allows for an authentication and authorization procedure, where necessary.
    \end{itemize}

    \item \textbf{A2} - Metadata are accessible, even when the software is no longer available.
\end{itemize}

\textbf{I: Software interoperates with other software by exchanging data and/or metadata, and/or
through interaction via application programming interfaces (APIs), described through
standards.}

\begin{itemize}
    \item \textbf{I1} - Software reads, writes and exchanges data in a way that meets domain-relevant community standards.
    \item \textbf{I2} - Software includes qualified references to other objects.
\end{itemize}

\textbf{R: Software is both usable (can be executed) and reusable (can be understood, modified, built
upon, or incorporated into other software).}

\begin{itemize}
    \item \textbf{R1} - Software is described with a plurality of accurate and relevant attributes.

    \begin{itemize}
        \item \textbf{R1.1} - Software is given a clear and accessible license.
        \item \textbf{R1.2} - Software is associated with detailed provenance.
    \end{itemize}

    \item \textbf{R2} - Software includes qualified references to other software.
    \item \textbf{R3} - Software meets domain-relevant community standards.
\end{itemize}

\subsection{Quality Attributes and the FAIR4RS principles}

In this subsection, the relation or link between the FAIR4RS principles and the Quality Attributes
show in Appendix \ref{appendix_qa} is represented in the following table:



% \textcolor{red}{DG: To be developed further once the quality metrics and criteria is clear. The scope of this section is not
% defining metadata for software, but stating how capturing these metadata may help addressing some of the quality metrics.
% For example, capturing schema:keywords may help having a complete description (not the greatest example, I know).}
